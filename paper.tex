\documentclass[10pt, letterpaper]{article}

\usepackage{amsfonts}
\usepackage{amsthm}
\usepackage{amsmath}

%%%%%%%%%%%%%%%%%%%%%%% Title info %%%%%%%%%%%%%%%%%%%%%%%%%%%%%%%%%%

\author{Kiran Vodrahalli, Evelyn Ding, Albert Lee}
\title{Word Sense Disambiguation: Pipelining with Supervised LSA}
\date{May 8, 2014}

\begin{document}

	\maketitle
	
	\section{Introduction}
	\subsection{Problem Statement}
	  Multiple word senses

	  ex. bank: "financial institution" or "river bank" or "count on something happening" or "supply/stock held in supply for future use" ...

	  Wrong sense can drastically alter meaning

	  Objective: given context, determine which sense of the word is intended

	  First proposed by Weaver in 1949 as computational task in the early days of MT

	  One of the oldest open problems in computational linguistics 

	  AI-Complete Problem

	  Example

	  "Paintings, drawings and sculpture from every period of art during the last 350 years will be on display ranging from a Tudor portrait to contemporary British art."
	  vs.
	  "Through casual meetings at cafe's, the artists drew together to form a movement in protest against the waste of war, against nationalism and against everything pompous, conventional or boring in the art of the Western world."


	  the creation of beautiful or significant things 
      vs. 
	  the products of human creativity; works of art collectively 
	\section{Previous Work}
	knowledge based

	unsupervised corpus-based

	supervised corpus-based
	\subsection{Knowledge-based}
	Lesk's algorithm:
two words W1 and W2 with multiple word senses
for each sense i in W1
for each sense j in W2
calculate overlap of i and j in dictionary definitions
maximize overlap i and j to determine word senses
Semantic similarity:
calculate path between words in wordnet for each word sense with other words in the context
minimize total semantic path to select word sense
context can vary from one sentence to length of document
	
	\subsection{Unsupervised Corpus-based}
	 Token-based discrimination: cluster contexts in which a given target word appears
	 Cluster 1: The line was occupied. 
	 The operator came into the line abruptly.

	 Cluster 2: The line was really long and it took forever to be served. 
 	 I stood in the queue for about 10 minutes

 	 Use parallel corpus across language to infer sense distinctions across the language
	 Je vais prendre ma propre décision.
	 vs.
	 Je vais prendre ma propre voiture.
	\subsection{Supervised Corpus-based}
	 Use annotated corpus to build model
 	 ex. to know: 1) be aware of piece of information
                              I want to know who is winning.
                              I know that the president lied.
                             2) know person
                             She doesn't know the composer.
                             Do you know my sister?
	 Use context of how word usually appears (bigram/trigram models)
 	 if (feature) then word sense
 	\subsection{LSA -- Past Efforts}
	\section{Data}
	\section{Algorithms}
	 Part of speech tagging and bigram models
 	 Use for subset of data that has high predictability
 	 Supervised LSA for the rest of the data

 	 Other methods had low accuracy 
 	\subsection{Part-of-Speech}
 	\subsection{Semantic Knowledgebase}
	\subsection{Supervised LSA}
	\section{Results}
	\section{Conclusions}
	\section{Future Work}
	\section{Citations} 

\end{document}
